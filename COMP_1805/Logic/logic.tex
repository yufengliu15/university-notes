\documentclass[11pt]{article}
\usepackage[utf8]{inputenc}	% Para caracteres en español
\usepackage{amsmath,amsthm,amsfonts,amssymb,amscd}
\usepackage{multirow,booktabs}
\usepackage[table]{xcolor}
\usepackage{fullpage}
\usepackage{lastpage}
\usepackage{enumitem}
\usepackage{fancyhdr}
\usepackage{mathrsfs}
\usepackage{wrapfig}
\usepackage{setspace}
\usepackage{calc}
\usepackage{multicol}
\usepackage{cancel}
\usepackage[retainorgcmds]{IEEEtrantools}
\usepackage[margin=3cm]{geometry}
\usepackage{amsmath}
\newlength{\tabcont}
\setlength{\parindent}{0.0in}
\setlength{\parskip}{0.05in}
\usepackage{empheq}
\usepackage{framed}
\usepackage[most]{tcolorbox}
\usepackage{xcolor}
\colorlet{shadecolor}{orange!15}
\parindent 0in
\parskip 12pt
\geometry{margin=1in, headsep=0.25in}
\theoremstyle{definition}
\newtheorem{defn}{Definition}
\newtheorem{reg}{Rule}
\newtheorem{exer}{Exercise}
\newtheorem{note}{Note}
% change with chapter
\newcommand\chaptercounter{1}
\setcounter{section}{\chaptercounter-1}
\begin{document}
\title{Chapter \chaptercounter}
\thispagestyle{empty}

\begin{center}
{\LARGE \bf Logic}\\
{\large COMP 1805}\\
Winter 2023
\end{center}

Understanding how statements work. Foundations are known as propositions (a declarative sentence that is either true or false). Math equations involving variables are not propositions(within reason).

\section{Operations}
\subsection{Negation}
Negation: if $p$ is a proposition.

Negation of $p$, denoted by $\neg p $ is statement: It is not the case that p
\begin{center}
	\begin{tabular}{ |c|c| } 
 		\hline
 		$P$ & $\neg p$ \\
		\hline
 		T & F \\
 		F & T \\
 		\hline
	\end{tabular}
\end{center}
\subsection{AND}
If $p$ and $q$ are propositions. Proposition $p$ and $q$ are denoted by $p \land q$, os a statement which is true if both $p$ and $q$ are true, otherwise false. "$\land$" is called conjunction
\begin{center}
	\begin{tabular}{ |c|c|c| } 
 		\hline
 		$P$ & $Q$ & $p \land q$\\
		\hline
 		T & T  & T\\
 		T & F & F\\
		F & T & F\\
		F & F & F \\
 		\hline
	\end{tabular}
\end{center}
\subsection{OR}
if $p$ and $q$ are propositions. Disjunction of $p$ and $q$, denoted by $p \lor q$ is the proposition $p$ or $q$.
\begin{center}
	\begin{tabular}{ |c|c|c| } 
 		\hline
 		$P$ & $Q$ & $p \lor q$\\
		\hline
 		T & T  & T\\
 		T & F & T\\
		F & T & T\\
		F & F & F \\
 		\hline
	\end{tabular}
\end{center}

XOR is the same as OR, only that both cannot be True (if they are both the same bool then False). Denoted by $p \oplus q$

\subsection{Implication}
If $p$ and $q$ are two propositions. Proposition $p \rightarrow q$ is statement of "if $p$ then $q$" and its truth value is False if $p$ is True and $q$ is False, otherwise True.
\begin{center}
	\begin{tabular}{ |c|c|c|} 
 		\hline
 		$P$ & $Q$ & $p \rightarrow q$  \\
		\hline
 		T & T  & T \\
 		T & F & F \\
		F & T & T \\
		F & F & T \\
 		\hline
	\end{tabular}
\end{center}
Converse of $p \rightarrow q$: $q \rightarrow p$ \\
Contraposition of $p \rightarrow q$: $q \rightarrow \neg p$ \\
Reverse of $p \rightarrow q$: $\neg p \rightarrow \neg q$

\begin{defn}
When two compound propositions have the same truth value for each assignment of values to the atomic proposition we say they are equivalent. 
\end{defn}
\subsection{Bi-implication (if and only if)}
if $p$ and $q$ are two propositions. Proposition $p \longleftrightarrow q$ is statement of "$p$ if and only if$q$" and its truth value is true if $p$ and $q$ have same truth value, otherwise false. 
\begin{center}
	\begin{tabular}{ |c|c|c|} 
 		\hline
 		$P$ & $Q$ & $p \longleftrightarrow q$  \\
		\hline
 		T & T  & T \\
 		T & F & F \\
		F & T & F \\
		F & F & T \\
 		\hline
	\end{tabular}
\end{center}
\end{document}





































