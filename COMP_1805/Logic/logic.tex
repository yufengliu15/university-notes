\documentclass[11pt]{article}
\usepackage[utf8]{inputenc}	% Para caracteres en español
\usepackage{amsmath,amsthm,amsfonts,amssymb,amscd}
\usepackage{multirow,booktabs}
\usepackage[table]{xcolor}
\usepackage{fullpage}
\usepackage{lastpage}
\usepackage{enumitem}
\usepackage{fancyhdr}
\usepackage{mathrsfs}
\usepackage{wrapfig}
\usepackage{setspace}
\usepackage{calc}
\usepackage{multicol}
\usepackage{cancel}
\usepackage[retainorgcmds]{IEEEtrantools}
\usepackage[margin=3cm]{geometry}
\usepackage{amsmath}
\newlength{\tabcont}
\setlength{\parindent}{0.0in}
\setlength{\parskip}{0.05in}
\usepackage{empheq}
\usepackage{framed}
\usepackage[most]{tcolorbox}
\usepackage{xcolor}
\colorlet{shadecolor}{orange!15}
\parindent 0in
\parskip 12pt
\geometry{margin=1in, headsep=0.25in}
\theoremstyle{definition}
\newtheorem{defn}{Definition}
\newtheorem{reg}{Rule}
\newtheorem{exer}{Exercise}
\newtheorem{note}{Note}
\newtheorem{eg}{Example}
% change with chapter
\newcommand\chaptercounter{1}
\setcounter{section}{\chaptercounter-1}
\begin{document}
\title{Chapter \chaptercounter}
\thispagestyle{empty}

\begin{center}
{\LARGE \bf Propositional Logic}\\
{\large COMP 1805}\\
Jan 09 2023
\end{center}

Understanding how statements work. Foundations are known as propositions (a declarative sentence that is either true or false). Math equations involving variables are not propositions(within reason).

Atomic propositions are the simplest things present in a proposition. They are the individuals that vary in true or false values. 
\section{Operations}
\subsection{Negation}
Negation: if $p$ is a proposition.

Negation of $p$, denoted by $\neg p $ is statement: It is not the case that p
\begin{center}
	\begin{tabular}{ |c|c| } 
 		\hline
 		$P$ & $\neg p$ \\
		\hline
 		T & F \\
 		F & T \\
 		\hline
	\end{tabular}
\end{center}
\subsection{AND}
If $p$ and $q$ are propositions. Proposition $p$ and $q$ are denoted by $p \land q$, os a statement which is true if both $p$ and $q$ are true, otherwise false. "$\land$" is called conjunction
\begin{center}
	\begin{tabular}{ |c|c|c| } 
 		\hline
 		$P$ & $Q$ & $p \land q$\\
		\hline
 		T & T  & T\\
 		T & F & F\\
		F & T & F\\
		F & F & F \\
 		\hline
	\end{tabular}
\end{center}
\subsection{OR}
if $p$ and $q$ are propositions. Disjunction of $p$ and $q$, denoted by $p \lor q$ is the proposition $p$ or $q$.
\begin{center}
	\begin{tabular}{ |c|c|c| } 
 		\hline
 		$P$ & $Q$ & $p \lor q$\\
		\hline
 		T & T  & T\\
 		T & F & T\\
		F & T & T\\
		F & F & F \\
 		\hline
	\end{tabular}
\end{center}

XOR is the same as OR, only that both cannot be True (if they are both the same bool then False). Denoted by $p \oplus q$

\subsection{Implication}
If $p$ and $q$ are two propositions. Proposition $p \rightarrow q$ is statement of "if $p$ then $q$" and its truth value is False if $p$ is True and $q$ is False, otherwise True.
\begin{center}
	\begin{tabular}{ |c|c|c|} 
 		\hline
 		$P$ & $Q$ & $p \rightarrow q$  \\
		\hline
 		T & T  & T \\
 		T & F & F \\
		F & T & T \\
		F & F & T \\
 		\hline
	\end{tabular}
\end{center}
Converse of $p \rightarrow q$: $q \rightarrow p$ \\
Contraposition of $p \rightarrow q$: $q \rightarrow \neg p$ \\
Reverse of $p \rightarrow q$: $\neg p \rightarrow \neg q$

\begin{defn}
When two compound propositions have the same truth value for each assignment of values to the atomic proposition we say they are equivalent. 
\end{defn}
\subsection{Bi-implication (if and only if)}
if $p$ and $q$ are two propositions. Proposition $p \longleftrightarrow q$ is statement of "$p$ if and only if$q$" and its truth value is true if $p$ and $q$ have same truth value, otherwise false. 
\begin{center}
	\begin{tabular}{ |c|c|c|} 
 		\hline
 		$P$ & $Q$ & $p \longleftrightarrow q$  \\
		\hline
 		T & T  & T \\
 		T & F & F \\
		F & T & F \\
		F & F & T \\
 		\hline
	\end{tabular}
\end{center}
\subsection{Priorties (like bedmas)} 
\begin{enumerate}
\item $\neg$
\item $\land$
\item $\lor$
\item $\oplus$
\item $\leftarrow$
\item $\longleftrightarrow$
\end{enumerate}

\section{Truth Tables}
\textbf{Prove} that  $\neg q \rightarrow \neg p \equiv \neg p \lor q$

\begin{proof} $\neg q \rightarrow \neg p \equiv \neg p \lor q$
\begin{center}
	\begin{tabular}{ |c|c|c|c|} 
 		\hline
 		$p$ & $q$ & $p \rightarrow q$ & $\neg p \lor q$ \\
		\hline
 		T & T  & T & T\\
 		T & F & F & F\\
		F & T & T & T \\
		F & F & T & T\\
 		\hline
	\end{tabular}
\end{center}
\end{proof}

Build the truth table of proposition $(p \lor \neg q) \rightarrow (p \land q)$
\begin{eg}
 \begin{center}
	\begin{tabular}{ |c|c|c|c|c|} 
 		\hline
 		$p$ & $q$ & $p \lor \neg q$ & $p \land q$ & $(p \lor \neg q) \rightarrow (p \land q)$\\
		\hline
 		T & T  & T & T & T\\
 		T & F & T & F& F\\
		F & T & F & F & T\\
		F & F & T & F & F\\
 		\hline
	\end{tabular}
\end{center}
\end{eg}
Sometimes truth value doesn't depend on the other truth values: the compound proposition is always true or always false, regardless of the truth assignments of the propositions. For example, $p \lor \neg p$ is always true, regardless of whether $p$ is true or false. This is known as \textbf{tautology}. \\

On the other hand, $p \land \neg p$ is always false, regardless of whether $p$ is true or false. This is known as a \textbf{contradiction}. \\

If a statement is neither of these, then it is known as a \textbf{contingency}. 

\section{Logical Equivalences}
\begin{reg}
Fundamental laws in logical operations. Let $T$ represent True and $F$ represent False
\begin{enumerate}
 \item Identity Law: $p \land T \equiv p$ and $p \lor F \equiv p$
\item Domination Law: $p \lor T \equiv T$ and $p \land F \equiv F$ 
\item Idempotent Law: $p \lor p \equiv p$ and $p \land p \equiv p$ 
\item Commutative Law: $p \lor q \equiv q \lor p$ and $p \land q \equiv q \land p$
\item Associative Law: $(p \lor q) \lor r \equiv p \lor (q \lor r)$ and $(p \land q) \land r \equiv p \land (q \land r)$ 
\item Distributive Law: $p \lor (q \land r) \equiv (p \lor q) \land (p \lor r)$ and $p \land (q \lor r) \equiv (p \land q) \lor (p \land r)$
\item Absorption Law: $p \lor (p \land q) \equiv p$ and $p \land (p \lor q) \equiv p$
\item De Morgan's Law: $\neg (p \land q) \equiv \neg p \lor \neg q $ and $\neg (p \lor q) \equiv \neg p \land \neg q$
\item Implication Equivalence: $p \rightarrow q \equiv \neg p \lor q$
\item Biconditional Equivalence: $p \longleftrightarrow q \equiv (p \rightarrow q) \land (q \rightarrow p)$
\end{enumerate}
\end{reg}
\end{document}





































