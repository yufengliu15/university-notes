\documentclass[11pt]{article}
\usepackage[utf8]{inputenc}	% Para caracteres en español
\usepackage{amsmath,amsthm,amsfonts,amssymb,amscd}
\usepackage{multirow,booktabs}
\usepackage[table]{xcolor}
\usepackage{fullpage}
\usepackage{lastpage}
\usepackage{enumitem}
\usepackage{fancyhdr}
\usepackage{mathrsfs}
\usepackage{wrapfig}
\usepackage{setspace}
\usepackage{calc}
\usepackage{multicol}
\usepackage{cancel}
\usepackage[retainorgcmds]{IEEEtrantools}
\usepackage[margin=3cm]{geometry}
\usepackage{amsmath}
\newlength{\tabcont}
\setlength{\parindent}{0.0in}
\setlength{\parskip}{0.05in}
\usepackage{empheq}
\usepackage{framed}
\usepackage[most]{tcolorbox}
\usepackage{xcolor}
\colorlet{shadecolor}{orange!15}
\parindent 0in
\parskip 12pt
\geometry{margin=1in, headsep=0.25in}
\theoremstyle{definition}
\newtheorem{defn}{Definition}
\newtheorem{reg}{Rule}
\newtheorem{exer}{Exercise}
\newtheorem{note}{Note}
\newtheorem{eg}{Example}
% change with chapter
\newcommand\chaptercounter{1}
\setcounter{section}{\chaptercounter -1}
\begin{document}
\title{Chapter \chaptercounter}
\thispagestyle{empty}

\begin{center}
{\LARGE \bf Chapter {\chaptercounter \hspace{0}} Lecture Notes}\\
{\large MATH 1104}\\
Winter 2023
\end{center}

\section{System of Linear Equations}
Some definitions to start with

\begin{defn}
A linear equation in variables $x_1, x_2, \cdots x_n$ is an equation that can be written in the form $a_1 x_1 + a_2 x_2 + \cdots + a_n x_n = b$, where $a_1, a_2, \cdots a_n \in \mathbb{R}$
\end{defn}

\begin{defn}
A system of linear equations (or linear system) is a collection of one or more linear equations involving the same variables , $x_1,x_2, \cdots, x_n$
$$a_{1_1} x_1 + a_{1_2} x_2 + \cdots + a_{1_n} x_n = b_1$$
$$a_{m_1} x_1 + a_{m_2} x_2 + \cdots + a_{m_n} x_n = b_m$$
\end{defn}

\begin{defn}
A solution of the system is a list ($S_1, S_2, \cdots S_n$) of numbers that makes each equation true when $S_1, S_2, \cdots S_n$are substituted for $x_1, x_2, \cdots x_n$

\begin{eg}
\begin{align*}
x_1 -2x_2 &= -1 \\
-x_1 + 3x_2 &= 3
\end{align*}
A solution here would be $(3,2)$, where $x_1 = 3$ and $x_2 = 2$. Note that this satisfies BOTH equations.
\end{eg}
To draw the graph set all the variables but one to 0. Repeat for the remaining variables. Only works for linear equations

A system of linear equations has either 

\begin{enumerate}
 \item No solutions (Inconsistent) (Parallel)\\
 \item Unique solution (Consistent)\\
 \item Infinite solutions (Consistent) (Same equation)
\end{enumerate}

\begin{align*}
x_1 -2x_2 + x_3 &= 0 \\
2x_2 - 8x_3 &= 8 \\
-4x_1 + 5x_2 + 9x_3 &= -9
\end{align*}

A=$\begin{bmatrix}
 1 & -2 & 1 \\
 0& 2 & -8 \\
 -4 & 5 & 9
\end{bmatrix}$
This is a coefficient matrix

x = $\begin{bmatrix}
x_1 \\
x_2\\
x_3
\end{bmatrix}$
b = $\begin{bmatrix}
0 \\
8\\
9
\end{bmatrix}$

Combine the A and b

Fundamental matrix questions:
\begin{enumerate}
 \item Does a solution of a linear system exist? \\
 \item How many solutions does it have if it is consistent? \\
 \end{enumerate}

\begin{defn}
The set of all solutions is called the solution set of the linear system. 
\end{defn}

\begin{eg}
\begin{align*}
x_1 -2x_2 + x_3 &= 0 \\
2x_2 - 8x_3 &= 8 \\
-4x_1 + 5x_2 + 9x_3 &= -9
\end{align*}

\begin{align*}
x_1 - 2x_2 + 3x_3 &= 0 \\
x_2 - 4x_3 &= 4 \\
x_3 &= 3
\end{align*}
\end{eg}
The equation on the bottom is easier to solve than the other. But what if you want to solve the top? Convert the complex system into an equivalent simpler system. 

\begin{defn}
Two linear systems are equivalent if they have the same solution set. 
\end{defn}

There are 3 elementary operations.

\begin{eg}
\begin{align*}
x+y &= 3 \\
2x-y &= 4
\end{align*}

1. Replace one equation by the sum of itself and a multiple of another equation
$$2x - y+(x+y) = 4 + 3 $$
$$3x = 7 $$

2. Multiplying by a non-zero constant 

3. Interchanging 2 equations

Perform Gaussian Elimination. Review it perhaps.

\section{Elementary Row Operations}
\begin{enumerate}
\item Replacement ($R_1 + 3R_2$ means $R_1 \leftarrow R_1 + 3R_2$)
\item Interchange ($R_1 \leftrightarrow R_2$)
\item Scaling (multiply the row by a non zero constant)
\end{enumerate}

\begin{defn}
 A rectangular matrix is in Echelon Form or Row Echelon Form, if it has the following properties:
 
\begin{enumerate}
 \item All nonzero rows are above any row of all zeros
 \item Each leading entry of a row is in a column to the right of leading entry of the row above it
 \item All entries in a column below a leading entry are zeros 
\end{enumerate}

\begin{eg}
 $\begin{bmatrix}
0 & 0 & 0 \\
8 & 1 & 3\\
9 & 4 & 0
\end{bmatrix}$ is not in REF
\end{eg}
\end{defn}

\begin{defn}
Reduced Row Echelon Form is REF, but with the leading entry equal to 1. As well, each leading 1 is the only nonzero entry in its column. 
\end{defn}
\end{document}





































