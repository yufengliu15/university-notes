\documentclass[11pt]{article}
\usepackage[utf8]{inputenc}	% Para caracteres en español
\usepackage{amsmath,amsthm,amsfonts,amssymb,amscd}
\usepackage{multirow,booktabs}
\usepackage[table]{xcolor}
\usepackage{fullpage}
\usepackage{lastpage}
\usepackage{enumitem}
\usepackage{fancyhdr}
\usepackage{mathrsfs}
\usepackage{wrapfig}
\usepackage{setspace}
\usepackage{calc}
\usepackage{multicol}
\usepackage{cancel}
\usepackage[retainorgcmds]{IEEEtrantools}
\usepackage[margin=3cm]{geometry}
\usepackage{amsmath}
\newlength{\tabcont}
\setlength{\parindent}{0.0in}
\setlength{\parskip}{0.05in}
\usepackage{empheq}
\usepackage{framed}
\usepackage[most]{tcolorbox}
\usepackage{xcolor}
\colorlet{shadecolor}{orange!15}
\parindent 0in
\parskip 12pt
\geometry{margin=1in, headsep=0.25in}
\theoremstyle{definition}
\newtheorem{defn}{Definition}
\newtheorem{reg}{Rule}
\newtheorem{exer}{Exercise}
\newtheorem{note}{Note}
\newtheorem{eg}{Example}
% change with chapter
\newcommand\chaptercounter{1}
\setcounter{section}{\chaptercounter -1}
\begin{document}
\title{Chapter \chaptercounter}
\thispagestyle{empty}

\begin{center}
{\LARGE \bf Chapter {\chaptercounter \hspace{0}} Lecture Notes}\\
{\large MATH 1104}\\
Winter 2023
\end{center}

\section{System of Linear Equations}
Some definitions to start with

\begin{defn}
A linear equation in variables $x_1, x_2, \cdots x_n$ is an equation that can be written in the form $a_1 x_1 + a_2 x_2 + \cdots + a_n x_n = b$, where $a_1, a_2, \cdots a_n \in \mathbb{R}$
\end{defn}

\begin{defn}
A system of linear equations (or linear system) is a collection of one or more linear equations involving the same variables , $x_1,x_2, \cdots, x_n$
$$a_{1_1} x_1 + a_{1_2} x_2 + \cdots + a_{1_n} x_n = b_1$$
$$a_{m_1} x_1 + a_{m_2} x_2 + \cdots + a_{m_n} x_n = b_m$$
\end{defn}

\begin{defn}
A solution of the system is a list ($S_1, S_2, \cdots S_n$) of numbers that makes each equation true when $S_1, S_2, \cdots S_n$are substituted for $x_1, x_2, \cdots x_n$

\begin{eg}
\begin{align*}
x_1 -2x_2 &= -1 \\
-x_1 + 3x_2 &= 3
\end{align*}
A solution here would be $(3,2)$, where $x_1 = 3$ and $x_2 = 2$. Note that this satisfies BOTH equations.
\end{eg}
To draw the graph set all the variables but one to 0. Repeat for the remaining variables. Only works for linear equations

A system of linear equations has either 

\begin{enumerate}
 \item No solutions (Inconsistent) (Parallel)\\
 \item Unique solution (Consistent)\\
 \item Infinite solutions (Consistent) (Same equation)
\end{enumerate}

\begin{align*}
x_1 -2x_2 + x_3 &= 0 \\
2x_2 - 8x_3 &= 8 \\
-4x_1 + 5x_2 + 9x_3 &= -9
\end{align*}

A=$\begin{bmatrix}
 1 & -2 & 1 \\
 0& 2 & -8 \\
 -4 & 5 & 9
\end{bmatrix}$
This is a coefficient matrix

x = $\begin{bmatrix}
x_1 \\
x_2\\
x_3
\end{bmatrix}$
b = $\begin{bmatrix}
0 \\
8\\
9
\end{bmatrix}$

Combine the A and b

Fundamental matrix questions:
\begin{enumerate}
 \item Does a solution of a linear system exist? \\
 \item How many solutions does it have if it is consistent? \\
 \end{enumerate}

\end{document}





































